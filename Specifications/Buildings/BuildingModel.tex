
\chapter{Transient building response model}
% Use \titlerunning{Short Title} for an abbreviated version of your contribution title if the original one is too long
%\author{Ruben Baetens and Dirk Saelens}
%\authorrunning{R. Baetens, D. Saelens}
%\institute{Ruben Baetens \at K.U.Leuven, Kasteelpark Arenberg 40 bus 2447, BE-3001 Leuven (Heverlee) \email{ruben.baetens@bwk.kuleuven.be}
%\and Dirk Saelens \at K.U.Leuven, Kasteelpark Arenberg 40 bus 2447, BE-3001 Leuven (Heverlee) \email{dirk.saelens@bwk.kuleuven.be}}
%\maketitle

%\abstract{A numeric building model is developed in Modelica for integrated energy simulation.}

%\vspace{\baselineskip}

\section{Introduction}

\section{Building thermal response}

% From an engineering point of view, the building is an energy sink in response to the outdoor climate and to be kept at potentials set by its occupants and health standards.

In this section, we describe in detail the dynamic building model and its possibilities that are implemented in Modelica as part of the \textsc{ideas} platform. The building model allows simulation of the energy demand for heating and cooling of a multi-zone building, energy flows in the building envelope and interconnection with dynamic models of thermal and electrical building energy systems within the platform for comfort measures. 

The description is divided into the description of the \index{wall} model and the \index{zone} model. The window model and the model for ground losses are described more in detail as extend to the wall model.

The relevant material properties of the surfaces are complex functions of the surface temperature, angle and wavelength for each participating surface. The assumptions used frequently in engineering applications~\cite{Walton1983} are that [``each surface emits or reflects diffusely and is at a uniform temperature''] and [``the energy flux leaving a surface is evenly distributed across the surface and one-dimensional'']. 

\subsection{Parallel opaque layers}

The description of the thermal response of a \index{wall} or a structure of parallel opaque layers in general is structured similar to the different occurring processes, \ie, the heat balance of the exterior surface, the heat balance of the interior surface and the heat conduction between both surfaces.

\subsubsection{Exterior surface heat balance}

The heat balance of the exterior surface $k$ is determined as \\*
\begin{equation}
\qflow_{cd}^{\s} + \qflow_{cv}^{\s} + \qflow_{sw}^{\s} + \qflow_{lw,e}^{\s} = 0
\end{equation}
where $\qflow_{cd}^{\s}$ denotes the conductive heat flow into the wall, $\qflow_{cv}^{\s}$ heat transfer by convection, $\qflow_{sw}^{\s}$ shortwave absorption of direct and diffuse solar light, $\qflow_{lw,e}^{\s}$ longwave heat exchange with the environment including the sky.

\paragraph{Convection}
We define the exterior convective heat flow between the exterior surfaces and the outdoor air based on an convective heat transfer coefficient as \\*
\begin{gather}
\qflow_{cv}^{\s} = \boldsymbol{h}_{cv}^{\s} \left(\T_{db,e} - \T_{s}^{\s}\right) \label{eq:qcve} \\
\boldsymbol{h}_{cv}^{\s} = \widetilde{\max}\left\{5.01\ \boldsymbol{v}_{10}^{0.85},\,5.6 \, ; \, 0.1\right\} \ \textrm{W}/\textrm{m}^{2}\textrm{K} \label{eq:hce}
\end{gather}
where $\T_{db,e}$ is the outdoor dry-bulb air temperature, $\T_{s}^{\s}$ is the surface temperature and $\boldsymbol{h}_{cv}^{\s}$ is the exterior convective heat transfer coefficient as defined in eqn. \ref{eq:hce} where $\boldsymbol{v}_{10}$ (\metre\per\second) is the wind velocity in the undisturbed flow field at 10 meter above the ground. The stated correlation with $\boldsymbol{v}_{10}$ is introduced to account for buoyancy driven convection at low wind velocities and forced convection induced by increasing velocities, and is valid for a $\boldsymbol{v}_{10}$ range of \mbox{$\left[0.15,7.5\right]$ \metre\per\second}.\citep{Defraeye2011,Jurges1924}

\paragraph{Longwave radiation}
We define the longwave radiative heat flow between the exterior surface and the environment as \\*
\begin{equation}
\qflow_{lw,e}^{\s}=\sigma \varepsilon_{lw}^{\s} \left(\T_{s}^{\s,4} - F_{ce}^{\s} \T_{ce}^{4} - (1-F_{ce}^{\s}) \T_{db}^{4}\right)
\label{eq:qlwe}
\end{equation}
%\begin{equation}
%F_{ce}^{\s} = 
%\end{equation}
as derived from the Stefan-Boltzmann law wherefore $\sigma$ is the Stefan-Boltzmann constant, $\varepsilon_{lw}^{\s}$ is the longwave emissivity of the exterior surface, $F_{ce}^{\s}$ the radiant-interchange configuration factor between the surface and the celestial dome as defined in eqn. \ref{eq:Fce}, $\T_{s}^{\s}$ is the surface temperature and $\T_{db}$ and $\T_{ce}$ are the outdoor dry bulb and celestial dome temperature respectively.\citep{Stefan1879,Boltzmann1884,Mohr2008,Hamilton1952} The stated eqn. \ref{eq:qlwe} is derived under the assumption that the ground temperature equals the outdoor air temperature, and that we can treat the surrounding outer environment as a much larger enclosure compared to the surface area.

\paragraph{Shortwave radiation}
We define the absorbed shortwave solar irradiation by the exterior surface in relation to the incident irradiation as \\*
\begin{equation}
\qflow_{sw}^{\s}=\varepsilon_{sw}^{\s} \left( \boldsymbol{E}_{Sd}^{\s} + \boldsymbol{E}_{SD}^{\s} \right)
\end{equation}
where $\varepsilon_{sw}^{\s}$ is the shortwave absorption of the surface, and $\boldsymbol{E}_{Sd}^{\s}$ and $\boldsymbol{E}_{SD}^{\s}$ are respectively the diffuse and direct solar irradiation striking the depicted exterior surface as defined in eqns. \ref{eq:ESd,eq:ESD}. 

\subsubsection{Interior surface heat balance}
\label{sec:int}

The heat balance of the interior surface is determined as \\*
\begin{equation} \label{eq:intersurf}
\qflow_{cd}^{\s} + \qflow_{cv}^{\s} + \sum_{j \in \mathcal{J}^{\s}} \qflow_{sw,j}^{\s} + \sum_{j \in \mathcal{J}^{\s}} \qflow_{lw,j}^{\s} = 0
\end{equation}
where $\qflow_{cd}^{\s}$ denotes the heat flow into the wall, $\qflow_{cv}^{\s}$ te heat transfer by convection, $\qflow_{sw}^{\s}$ the shortwave absorption of direct and diffuse solar light entering the interior zone through windows and $\qflow_{lw}^{\s}$ the longwave heat exchange with the surrounding interior surfaces.

\paragraph{Convection}
We define the interior convective heat flow between the interior surfaces and zone air based on an convective heat transfer coefficient as \\*
\begin{gather}
\qflow_{cv}^{\s} = \boldsymbol{h}_{cv}^{\s} \left(\T_{db,i} - \T_{s}^{\s}\right) \\
\boldsymbol{h}_{cv}^{\s} = \widetilde{\max}\left\{1,\,n_{1}^{\s} D^{\s n_{2}^{\s}} \left|\T_{db,i}-\T_{s}^{\s}\right|^{n_{3}^{\s}} ;\, 0.1 \right\}
\label{eq:hcvi}
\end{gather}
similar to eqn. \ref{eq:qcve} with $\T_{db,i}$ the indoor dry-bulb temperature, $\T_{s}^{\s}$ the surface temperature and $\boldsymbol{h}_{cv}^{\s}$ the interior natural convective heat transfer coefficient. As the room-side heat transfer coefficient couples the room air temperature to the temperature of building components, detailed computation of its value is important. Therefore we define $\boldsymbol{h}_{cv}^{\s}$ as in eqn. \ref{eq:hcvi} where $D^{\s}$ is the characteristic length of the surface, $T_{db}$ is the indoor air temperature, $T_{s}^{\s}$ the surface temperature and $n_{x}^{\s}$ are coefficients correlating the occurring buoyancy forces and the heat transfer. These parameters $\{n_{1}^{\s},n_{2}^{\s},n_{3}^{\s}\}$ are set identical to $\{1.823,-0.121,0.293\}$ for vertical surfaces, $\{2.175,-0.076,0.308\}$ for horizontal surfaces with enhanced convection, \ie, with a heat flux in the same direction as the buoyancy force, and $\{2.72,-,0.13\}$ for horizontal surfaces with reduced convection, \ie, with a heat flux in the opposite direction as the buoyancy force as defined by Khalifa \etal and Awbi \etal\citep{Khalifa2001,Awbi1999} Note that the interior natural convective heat transfer coefficient is only described as function of the temperature difference. An overview of more detailed correlations including the possible higher air velocities due to mechanical ventilation can be found in literature but are not implemented.\citep{Beausoleil-Morrison2000}

\paragraph{Longwave radiation}
Longwave radiation between two internal surface $k$ and $j$ can be described by a thermal circuit formulation as\citep{Buchberg1954,Buchberg1955,Oppenheim1956} \\*
\begin{gather}
\qflow_{lw,j}^{\s}= \sigma \left(\frac{\varsigma_{lw}^{\s}}{\varepsilon_{lw,j}} + \frac{1}{F_{j}^{\s}} + \frac{A^{\s}}{\sum A_{j \in \mathcal{J}^{\s}}}\right)^{-1} \left(\T_{s}^{\s,4}-\T_{s,j}^{4}\right) \\
F_{j}^{\s} = \int_{\Omega_{j}} \cos \theta_{p} \cos \theta_{x} \pi^{-1} S_{j}^{\s,-2} d\Omega_{j}
\label{eq:Fjk}
\end{gather}
wherefore $\varepsilon_{lw}^{\s}$ and $\varepsilon_{lw,j}$ are the longwave emissivities of the surfaces, $F_{j}^{\s}$ is radiant-interchange configuration factor between these surfaces, $A^{\s}$ and $A_{j}$ are the areas of surfaces $k$ and $j$ respectively wherefore $\mathcal{J}^{\s}$ is the set of all surfaces $j$ surrounding surface $k$, $\sigma$ is the Stefan-Boltzmann constant and $T_{s}^{\s}$ and $T_{s,j}$ are the respective surface temperatures.\citep{Stefan1879,Boltzmann1884,Hamilton1952,Mohr2008} \\*
%reformulating this to something computable
The above description of longwave radiation for a room or thermal zone results in the necessity of a very detailed input, \ie, the configuration between all surfaces needs to be described by their shape, position and orientation in order to define $F_{j}^{k}$ as eqn. \ref{eq:Fjk} which introduces main difficulties for windows and internal heat gain sources in the zone of interest. Simplification is achieved by means of a delta-star transformation and by definition of a (fictive) radiant star node in the zone model.\citep{Kenelly1899} Literature shows that the overall model is not significantly sensitive to this assumption.\citep{Liesen1997} The heat exchange by longwave radiation between surface $s_{i}$ and the radiant star node in the zone model can be described as \\*
\begin{equation}
\sum_{j \in \mathcal{J}^{\s}} \qflow_{lw,j}^{\s}= \sigma \left(\frac{\varsigma_{lw}^{\s}}{\varepsilon_{lw,j}} + \frac{A^{\s}}{\sum A_{j \in \mathcal{J}^{\s}}}\right)^{-1} \left(\T_{s}^{\s,4}-\T_{rs}^{4}\right)
\end{equation}
where $\varepsilon_{x}$ is the emissivity of surface $x$, $\varsigma_{lw}$ equals $1-\varepsilon_{x}$, $A_{x}$ is the area of surface, $\sum_{j}A_{x}$ is the sum of areas for all surfaces $s_{\textrm{j}}$ of the thermal zone, $\sigma$ is the Stefan-Boltzmann constant~\cite{Mohr2008} and $T_{\textrm{si}}$ and $T_{\textrm{si}}$ are the temperatures of surfaces $x$ and the radiant star node  respectively. 

\paragraph{Shortwave radiation}
Absorption of shortwave solar radiation on the interior surface is handled equally as for the outside surface. Determination of the receiving solar radiation on the interior surface after passing through windows is dealt with in the zone model.

\subsubsection{Wall conduction process}

%add here also the posibility of transfer functions

For the purpose of dynamic building simulation, the partial differential equation of the continuous time and space model of heat transport through a solid is most often simplified into ordinary differential equations with a finite number of parameters representing only one-dimensional heat transport through a construction layer. Within this context, the wall is modeled with lumped elements, i.e. a model where temperatures and heat fluxes are determined from a system composed of a sequence of discrete resistances and capacitances $R_{n+1}$, $C_{n}$. The number of capacitive elements $n$ used in modeling the transient thermal response of the \index{wall} denotes the order of the lumped capacitance model. \\*
\begin{equation}
\Qflow_{\textrm{net},x} = \frac{\partial \T_{\textrm{ci}}}{\partial t}C_{x} = \sum_{i}^{n} \Qflow_{\textrm{res},x} + \Qflow_{\textrm{source},x}
\end{equation}
where $\dot{Q}_{\textrm{net},x}$ is the added energy to the lumped capacity, $T_{\textrm{ci}}$ is the temperature of the lumped capacity, $C_{\textrm{ci}}$ is the thermal capacity of the lumped capacity equal to $\rho_{x}c_{x}d_{\textrm{c}}A_{x}$ for which $\rho_{x}$ denotes the density and $c_{x}$ is the specific heat capacity of the material and $d_{\textrm{ci}}$ the equivalent thickness of the lumped element, where $\dot{Q}_{\textrm{res},x}$ the heat flux through the lumped resistance and $R_{\textrm{ri}}$ is the total thermal resistance of the lumped resistance equal to $d_{\textrm{ri}}\left(\lambda_{x}A_{x}\right)^{-1}$ for which $d_{\textrm{ri}}$ denotes the equivalent thickness of the lumped element and where $\dot{Q}_{\textrm{source}}$ are internal thermal source, \eg from embedded systems.

%RC-model makes it easy to implement / easy for adding solar gains and radiation / easy for including local sources, floor heating, ... though might result in longer simulation times.

%Studies on the optimal order reduction for lumped construction elements in thermal building models can be found in literature~\cite{Tindale1993,Gouda2000,Gouda2002,Wang2006,Xu2007}, where optimization towards reduction is performed through comparison of zone air temperatures or comparison of Bode plots~\cite{Bode1945} on magnitude and phase for the low-order and a high-order lumped element. The general conclusion found towards model accuracy and computational efficiency depict that 1st-order lumped elements do not seem to be able to deal with radiation on the surfaces whereas 2nd-order lumped elements, i.e. based on two capacities and three resistances, give minimal loss of accuracy compared to high-order reference models for a limited computational effort. Both light and medium constructions~\cite{ASHRAE2009} show high accuracy if a 2nd-order lumped element is used and little improvements can be achieved through optimization on nodal placement~\cite{Tindale1993,Gouda2002,Wang2006,Xu2007} whereas a higher order thermal network should be used for heavy constructions~\cite{ASHRAE2009} when the dynamics of the system are of concern as significant errors remain for simplified models at low frequency~\cite{Wang2006,Xu2007,Masy2008}.

%The model has a provision for including a temperature coefficient $f_{\lambda,c}$ to modify the thermal conductivity. The general description for the temperature dependency of the material thermal conductivity $\lambda$ is $\lambda_{0} + f_{\lambda,c}\left[T_{C}-T_{0}\right]$ where $T_{0}$ is the temperature for which the standard input thermal conductivity is defined at standard temperature and pressure (STP) conditions. If $f_{\lambda,c}$ is not defined, no temperature dependence is taken into account and set to unity.

% The model has a provision for including a temperature coefficient to modify the thermal capacity to take into account phase change materials.

%The Detailed version determines the number of nodes in each layer of the surface based on the Fourier stability criteria. The node thicknesses are normally selected so that the time step is near the explicit solution limit in spite of the fact that the solution is implicit. For very thin, high conductivity layers, a minimum of two nodes is used. This means two half thickness nodes with the node temperatures representing the inner and outer faces of the layer. All thicker layers also have two half thickness nodes at their inner and outer faces. These nodes produce layer interface temperatures. ====>>> This can not be done here as the solver has a variable step size solver, no ?!

\subsection{Windows}

The thermal model of a \index{window} is similar to the model of an exterior wall but includes the absorption of solar irradiation by the different glass panes, the presence of gas gaps between different glass panes and the transmission of solar irradiation to the adjacent indoor zone.

\paragraph{Gap heat transfer}
The total convective and longwave heat transfer through thin gas gaps as present in modern glazing systems is described as \\*
\begin{gather}
\qflow_{cd} = \mathbf{Nu}_{g} \frac{\lambda_{g}}{d_{g}} \left(\T_{s}^{\s}-\T_{s}^{(j)}\right) + \sigma \frac{\varepsilon_{k} \varepsilon_{j}}{1-\varsigma_{k}\varsigma_{j}} \left(\T_{s,k}^{4}-\T_{s,j}^{4}\right) \\
\mathbf{Nu}_{g} = n_{\textrm{i,1}} \mathbf{Gr}_{g}^{n_{\textrm{i,2}}} \\
\mathbf{Gr}_{g} = g \beta \rho^{2} d_{g}^{3} \mu^{-2} \left(\T_{\textrm{si}}-\T_{\textrm{sj}}\right)
\end{gather}
where $A_{x}$ is the glazing surface, $d_{x}$ is the gap width, $\textrm{Nu}_{x}$ is the Nusselt number of the gas, $\varepsilon_{x}$ is the longwave emissivity of the surfaces, $\varsigma_{x}$ equals $1-\varepsilon_{x}$ and $T_{\textrm{si}}$ is the surface temperature. The Nusselt number of the present gas in the gap describing the ratio of convective to conductive heat transfer is generally described is where $\textrm{Gr}_{x}$ is the Grashof number approximating the ratio of buoyancy to viscous force acting on the window gap gas, $g$ is the gravitational acceleration, $\beta$ is the coefficient of thermal expansion, $\rho$ is the gas density, $\mu$ is the gas viscosity and $n_{\textrm{i,:}}$ are correlation coefficients. These parameters $\{n_{\textrm{i,1}},n_{\textrm{i,2}}\}$ are identical to $\{1.0,0\}$ for all $\textrm{Gr}_{x}$ below $7.10^{3}$, $\{0.0384,0.37\}$ for all $\textrm{Gr}_{x}$ between $10^{4}$ and $8.10^{4}$, $\{0.41,0.16\}$ for all $\textrm{Gr}_{x}$ between $8.10^{4}$ and $2.10^{5}$ and $\{0.0317,0.37\}$ for all $\textrm{Gr}_{x}$ above $2.10^{5}$. 

\paragraph{Shortwave optical properties}
The properties for absorption by and transmission through the glazing are taken into account depending on the angle of incidence of solar irradiation and are based on the output of the WINDOW 4.0 software~\cite{Lawrence1993,Finlayson1993} as validated by Arasteh~\cite{Arasteh1986} and Furler~\cite{Furler1988}. Within this software, the angular dependence of the optical properties is determined based on the model of Furler~\cite{Furler1991}. The reflectivity $r$ and  transmissivity $t$ are determined with the Fresnel equations~\cite{Fresnel1926} and Snell's Law of refraction~\cite{Snellius1627},  based on the relative refractive index n as follows \\*
\begin{gather}
r_{x} = 0.5\sin^{2}\Delta_{x} \sin^{-2}\Delta_{x} + 0.5\tan^{2}\Delta_{x}\tan^{-2}\Delta_{x} \\
\Delta_{x}=\xi_{x}-\xi'_{x} \mbox{ st. } \sin \xi_{x} = n\sin \xi'_{x}
\end{gather}
where $\Delta_{x}=\xi_{x}-\xi'_{x}$ wherefore counts that $\sin \xi_{x} = n\sin \xi'_{x}$. The resulting transmittance $T_{x}$ and reflectance $R_{x}$ for a single glass pane after multiple reflections is obtained from \\*
\begin{gather}
T_{x} = t_{0}^{2} \, \exp\left(-\alpha_{x} d_{x}/\cos \xi_{x} \right) \left(1-r_{x}^{2} \exp\left(-\alpha d_{x}/\cos \xi_{x}\right)\right)^{-1} \\
R_{x} = r_{x} \left(1+T_{x} \, \exp\left(-\alpha_{x} d_{x}/\cos \xi_{x}\right)\right)
\end{gather}
where $d_{x}$ is the thickness of the pane and $\alpha_{x}$ is absorption coefficient. The total transmittance $T_{x}$ and the absorptances $A_{\textrm{i,:}}$ for multipane windows are retrieved using iterative equations taking into account the multiple internal reflections within the glazing system.

The resulting output from WINDOW 4.0 ~\cite{Lawrence1993} depicts an array of the transmittances $T$ through the window and the absorptances $A_{x}$ for each glass pane $x$ for $\xi_{x}\in\left\{\left(k/18\right\}\right)$ with $k\in\left\{0,1,\ldots,9\right\}$. The same array input is used in other dynamic building simulation tools, e.g. in TRNSYS~\cite{Solar2009} where values for different angles are retrieved by means of linear interpolation. 

\subsection{Ground slabs}

The heat flow through building envelope constructions in contact with a \index{ground slab} is the same for the interior surface and the \index{wall} conduction process, but differs at the exterior surface in contact with the ground. As the heat transfer through the ground is 3-dimensional and defined by a large time lag, the exterior surface heat balance is generally approximated based on ISO 13370.

The total heat flow through the ground is given by \\*
\begin{equation}
\dot{Q}_{\textrm{net,i}} = L_{\textrm{S,i}} \left[ \bar{T}_{i} - \bar{T}_{e} \right] - L_{\textrm{pi,i}} \hat{T}_{i} \cos \gamma_{i} + L_{\textrm{pe,i}}(x) \hat{T}_{e} \cos \gamma_{e}
\end{equation}
where $L_{\textrm{S,i}}$ is the steady-state thermal coupling coefficient, $L_{\textrm{pi}}$ and $L_{\textrm{pe}}$ are the internal and external periodic thermal coupling coefficients respectively, $\bar{T}$ is the annual average temperature, $\hat{T}$ is the annual average temperature amplitude, and $\gamma_{i}$ and $\gamma_{e}$ determine the time lag of the heat flow cycle compared with that of the internal and external temperature respectively.

The steady-state and periodic thermal coupling coefficient area is determined as \\*
\begin{gather}
L_{\textrm{S,i}} = A_{x} \frac{\lambda_{g}}{0.457 B_{t} + d_{t} + 0.5 z}+ z P \frac{2 \lambda_{g}}{\pi z} \left[1+\frac{0.5 d_{t}}{d_{t}+z}\right] \ln \left[\frac{z}{d_{\textrm{t,i}}}+1\right] \\
L_{\textrm{pi,i}} = A_{x} \frac{\lambda_{\textrm{g}}}{d_{\textrm{t,i}}} \sqrt{\frac{2}{\left[ 1 + \frac{\delta}{d_{\textrm{t,i}}}\right]^{2} +1}} \; \mbox{and} \;
L_{\textrm{pe,i}} = 0.37 P_{x} \lambda_{\textrm{g}} \ln{\left[\frac{\delta}{d_{\textrm{t,i}}} + 1\right]}
\end{gather}
where $A_{x}$ is the wall area, $\lambda_{\textrm{g}}$ is the thermal conductivity of the unfrozen ground, $B_{\textrm{t,i}}(x)$ is the characteristic dimension of the floor, $d_{\textrm{t,i}}$ is the equivalent thickness of the wall construction, $z$ is the depth of the wall (i.e. floor) below ground level, $\delta$ is periodic penetration depth (i.e. the depth in the ground at which the temperature amplitude is reduced to $e^{-1}$ of that at the surface) and $P_{x}$ is the exposed perimeter of the wall. The angle $\gamma_{e}$ is determined as $2 \pi t / t_{\textrm{yr}} + \pi / 12 - \arctan{d_{\textrm{t,i}} / \left( d_{\textrm{t,i}} + \delta \right)}$ and $\gamma_{e}$ is determined as $2 \pi t / t_{\textrm{yr}} + \pi / 12 + 0.22 \arctan{\delta / \left( d_{\textrm{t,i}} + 1 \right)}$.

\subsection{Zones}

Consisting of both the convective as radiative calculation for determination of thermal comfort.

\subsubsection{Thermal response model}

Also the thermal response of a \index{zone} can be divided into a convective, longwave radiative and shortwave radiative process influencing both thermal comfort in the depicted zone as well as the response of adjacent wall structures.

%\paragraph{Convective}
The air within the zone is modeled based on the assumption that it is well-stirred, i.e. it is characterized by a single uniform air temperature. This is practically accomplished with the mixing caused by the air distribution system. The convective gains and the resulting change in air temperature $T_{a}$ of a single thermal zone can be modeled as a thermal circuit. The resulting heat balance for the air node can be described as \\*
\begin{equation}
\begin{split}
\frac{\partial T_{\textrm{a}}}{\partial t} c_{\textrm{a}} V_{\textrm{a}} & = \sum_{i}^{N} \dot{Q}_{\textrm{i,a}} + \sum_{i}^{n_{s}}R_{\textrm{ci}}^{-1}A_{x}\left[T_{\textrm{a}}-T_{\textrm{si}}\right] +\sum_{i}^{n_{z}}\dot{m}_{\textrm{a,z}}\left[h_{\textrm{a}}-h_{\textrm{a,z}}\right]  \\ & +\dot{m}_{\textrm{a,e}}\left[h_{\textrm{a}}-h_{\textrm{a,e}}\right] +\dot{m}_{\textrm{a,sys}}\left[h_{\textrm{a}}-h_{\textrm{a,sys}}\right]
\end{split}
\end{equation}
wherefore the specific air enthalpy $h_{\textrm{a}}$ is determined as $c_{\textrm{a}}\vartheta_{a}+\chi_{\textrm{a}} c_{\textrm{w}}\vartheta_{\textrm{a}} + \chi_{\textrm{a}} h_{\textrm{w,ev}}$ and where $T_{\textrm{a}}$ is the air temperature of the zone, $c_{\textrm{a}}$ is the specific heat capacity of air at constant pressure, $V_{\textrm{a}}$ is the zone air volume, $\dot{Q}_{\textrm{a}}$ is a convective internal load, $R_{\textrm{ci}}$ is the convective surface resistance of surface $x$, $A_{x}$ is the area of surface $x$, $T_{\textrm{si}}$ the surface temperature of surface $x$, $\dot{m}_{\textrm{a,z}}$ is the mass flow rate between zones, $\dot{m}_{\textrm{a,e}}$ is the mass flow rate between the exterior by natural infiltration, $\dot{m}_{\textrm{a,sys}}$ is the mass flow rate provided by the ventilation system, $\vartheta_{\textrm{a}}$ is the air temperature in degrees Celsius, $\chi_{\textrm{a}}$ is the air humidity ratio, $c_{w}$ is specific heat of water vapor at constant pressure and $h_{\textrm{w,ev}}$ is evaporation heat of water at 0 degrees Celsius. 

Infiltration and ventilation systems provide air to the zones, undesirably or to meet heating or cooling loads. The thermal energy provided to the zone by this air change rate can be formulated from the difference between the supply air enthalpy and the enthalpy of the air leaving the zone $h_{\textrm{a}}$. It is assumed that the zone supply air mass flow rate is exactly equal to the sum of the air flow rates leaving the zone, and all air streams exit the zone at the zone mean air temperature. The moisture dependence of the air enthalpy is neglected in most cases.

A multiplier for the zone capacitance $f_{\textrm{c,a}}$ is included. A $f_{\textrm{c,a}}$ equaling unity represents just the capacitance of the air volume in the specified zone. This multiplier can be greater than unity if the zone air capacitance needs to be increased for stability of the simulation. This multiplier increases the capacitance of the air volume by increasing the zone volume and can be done for numerical reasons or to account for the additional capacitances in the zone to see the effect on the dynamics of the simulation. This multiplier is constant throughout the
simulation and is set to 5.0 if the value is not defined. %add a source here, look in the PhD of Gabrielle Masy

%\paragraph{Longwave radiation}
The exchange of longwave radiation in a zone has been previously described in Sect.~\ref{sec:int} by Eq.~\ref{eq:intersurf} and further considering the heat balance of the interior surface. Here, an expression based on \emph{radiant interchange configuration factors} of \emph{view factors} is avoided based on a delta-star transformation and by definition of a \emph{radiant star temperature} $T_{\textrm{rs}}$. Literature~\cite{Liesen1997} shows that the overall model is not significantly sensitive to this assumption. This $T_{\textrm{rs}}$ can be derived from the law of energy conservation in the radiant star node as $\sum_{i}Q_{\textrm{i-rs}}$ must equal zero. Long wave radiation from internal sources are dealt with by including them in the heat balance of the radiant star node resulting in a diffuse distribution of the radiative source.

%\paragraph{Shortwave radiation}
Transmitted shortwave solar radiation is distributed over all surfaces in the zone in a prescribed scale. This scale is an input value which may be dependent on the shape of the zone and the location of the windows, but literature~\cite{Liesen1997} shows that the overall model is not significantly sensitive to this assumption.

