\documentclass{book}
\usepackage{breqn}
\bibliographystyle{ieeetr}

\begin{document}

\chapter{Climate}

In this section, we describe in detail the climate model and its possibilities that are implemented in Modelica as part of the IDEAS platform. Four external factors are to be known, i.e. external temperature and ground temperature for transient heat losses by conduction, sky temperature for long-wave radiation losses and short-wave gains on surfaces by solar irradiation.

\section{Weather data}
\label{chap:climwea}

The main weather parameters required for transient thermal building simulation are the ambient dry-bulb temperature $T_{db}(t)$, the outdoor relative humidity $\varphi_{e}(t)$, the wind speed $v_{10}(t)$, the diffuse horizontal solar radiation $E_{d,h}(t)$ and direct normal solar radiation$E_{D,\bot}(t)$. 

The Meteonorm system~\cite{Meteotest2008} is a comprehensive source of (all mentioned) weather data for engineering applications in Europe and this system is used within this context. For simulation, the retrieved data from the Meteonorm system are not used within the common formats of a test reference year *.try~\cite{NCDC1981,EC1985} as used in Europe or the formats of a typical meteorological years *.tm or *.tmy~\cite{NCDC1976b,NREC1995} and weather years for energy calculations *.wyec or *.wyec~\cite{ASHRAE1985} as used in the United States and Canada. These data formats are derived from hourly observations at a specific location by the national weather service or meteorological office and contain too little information for sub-hourly simulation, especially towards renewable energy generation by solar radiation.

From the retrieved data from the Meteonorm system, one more temperature needs to be determined. The long-wave radiative heat exchange of an exterior surface with a cloudy sky is calculated based on a sky temperature. This black-body sky temperature $T_{sky}(t)$ can be determined~\cite{Walton1983,Martin1984} as

\begin{equation}
T_{sky}(t)=T_{db}(t)\epsilon_{sky}(t)^{0.25}
\end{equation}

\begin{equation}
\epsilon_{sky}(t) = \epsilon_{0}(t)+\Delta\epsilon_{h}(t)+CCF(t)\left[1-\epsilon_{h}(t)-\Delta\epsilon_{h}(t)\right]
\end{equation}

where $\epsilon_{sky}(t)$ is the cloudy sky emissivity~\cite{Berdahl1982,Berdahl1984,Martin1984}, where $\epsilon_{0}(t)+\Delta\epsilon_{h}(t)$ is the clear sky emisivity $\epsilon_{clear}(t)$, $\Delta\epsilon_{h}(t)$ is a diurnal correction taking into account the difference in sky emissivity between day and night and CCF$(t)$ is the cloud cover factor. Both  $\epsilon_{clear}(t)$ and CCF$(t)$ are determined as polynomial fits on measurement data:

\begin{equation}
\epsilon_{0}(t)+\Delta\epsilon_{h}(t) = 0.711 + 0.0056\ \vartheta_{dew}(t) + 0.000073\ \vartheta_{dew}(t)^{2} + 0.013\ \cos h(t)
\end{equation}

\begin{equation}
CCF(t) = 1.0 + 0.024\ CC(t) - 0.0035\ CC(t)^{2} + 0.00028\ CC(t)^{3}
\end{equation}

where $h(t)$ is the hour angle, $\vartheta_{dew}(t)$ is the dew temperature and CC$(t)$ is the tenths cloud cover retrieved from Meteonorm~\cite{Kasten1979,Perraudeau1990}. 

\section{Solar radiation}
\label{chap:climsol}

The calculation of the direct and diffuse solar irradiation on a tilted surface requires determination of the position of the sun in the sky. Here, the zenith angle $\xi(t,x)$ of surface with inclination $i(x)$ and azimuth $a(x)$ are able to uniquely define the the solar radiation on a tilted surface based on the determination of the annual and daily solar cycle by means of solar time and declination.

Within the computational model, all solar irradiation calculations are handled in Commons.Meteo.Solar.RadSol which is built-in in each surface receiving solar radiation.

\subsection{Solar geometry}

The \emph{apparant solar time} $t_{sol}(t)$ expressed in seconds is based on daily apparent motion of the sun as seen from the earth. Solar noon is defined as the moment when the sun reaches the highest point in the sky. Solar time defined as

\begin{equation}
t_{sol}(t) = t_{std}(t) + 720\pi^{-1}\left[L_{std}-L_{loc}\right] + E_{t}(t)
\end{equation}

\begin{equation}
E_{t}(t) = -120\ e\sin M(t) + 60\tan^{2} \left(\epsilon/2\right) \sin \left(2M(t)+2\lambda_{p}\right)
\end{equation}

where $t_{std}(t)$ is the standard time of the time zone, $L_{std}$ is the reference meridian, $L_{loc}$ is the lcoal meridian and $E_{t}$ is the \emph{equation of time} defining the difference between solar noon and noon of local civil time, $M(t)$ is the mean anomaly relating to the position of the sun to the earth in a Kepler orbit, $\epsilon$ is the earth obliquity and $\lambda_{p}$ the ecliptic longitude of the periapsis, i.e. the closest approach of the earth to the sun.

Daylight saving time is taken into account within the simulation and corrects $t_{std}(t)$. Daylight saving time starts in the \emph{European Economic Community} on March $31-\left[(5y)/4+4\right]$ mod $7$ and ends on October $31-\left[(5y)/4+1\right]$ mod $7$ where $y$ denotes the year and mod denotes the remainder by division~\cite{VanGent2011}.

Before the zenith angle can be calculated, the declination $\delta$ and solar hour angle $h(t)$ is to be defined to fully specify the position of the sun as seen by an observer at a given time. Here, $\delta(t)$ depicts the angle between the solar beam and the equatorial plane, defined~\cite{Spencer1971} as 

\begin{equation}
\sin \delta(t) = \sin \epsilon \cos \left(2\pi \left(n(t)+10\right) n_{y}^{-1} \right)
\end{equation}

where $\epsilon$ is the earth obliquity, $n(t)$ is the one-based day number, i.e. 1 for January 1, and $n_{y}$ is the length in days of the earth revolution equal to 365.25 days. The correction of 10 days is required as winter solstice, i.e. when the apparent position of the sun in the sky as viewed from the Earth reaches its most northern extreme, occurs at December 21. 

The hour angle $h(t)$ depicts the angle between between the half plane of the Earth's axis and the zenith and the half plane of the Earth's axis and the given location, defined as

\begin{equation}
h(t) = 2\pi t_{sol}(t)\ 86400^{-1} - \pi
\end{equation}

where $t_{sol}(t)$ is solar time.

Based on $\delta(t)$ and $h(t)$, the zenith angle $\xi(t,x)$ of a surface with inclination $i(x)$ and azimuth $a(x)$ can be uniquely defined. The zenith angle of the sun to a surface is the angle between this surface normal and the sun's beam, and is derived from~\cite{Duffie1980,Iqbal1983}

\begin{dmath} 
\cos\xi(t,x)=\sin\delta(t)  \sin\varphi  \cos i(x)-\sin\delta(t)  \cos\varphi  \sin i(x) \cos a(x)+\cos\delta(t)  \cos\varphi  \cos i(x) \cos a(x) +\cos\delta(t)  \cos h(t)  \sin\varphi  \sin i(x)  \cos a(x) +\cos\delta(t)  \sin h(t) \sin i_{s} \sin a(x) 
\end{dmath}

where $\varphi$ is the latitude of the location defined positive for the northern hemisphere, $h(t)$ is the hour angle, $i(x)$ is the surface inclination defined as 0 for ceilings and $\pi/2$ for vertical walls, $a(x)$ is the surface azimuth defined as $-\pi/2$ if the surface outward normal points eastward and 0 if the normal points southward, and where $\delta(t)$ is the solar declination.

\subsection{Shortwave radiation on a tilted surface}

The total solar irradiation $E(t,x)$ on a arbitrary surface can be determined as the sum of the direct $E_{D}(t,x)$, diffuse $E_{d}(t,x)$ and reflected $E_{r}(t,x)$ radiation on the surface. 

\begin{equation}
E(t,x)=E_{D}(t,x)+E_{d}(t,x)+E_{r}(t,x)
\end{equation}

For a known profile of direct solar irradiation on a random surface, all three factors can be determined for another arbitrary surface s. Herefore, a profile of direct solar irradiation $E_{D,\bot}(t,x)$ perpendicular on the beam radiation is retrieved from Meteonorm and used as only input parameter. The calculation of other configurations besides normal to the solar beam is performed in the model. 

Different models for the determination of the diffuse radiation do exist based on an isotropic or anisotropic model of the sky dome. On account of the high importance of solar irradiation for the model\footnote{Solar radiation interacts e.g. with the building thermal response, heat generation by means of a thermal solar collector and power generation with a photovoltaic array}, a more detailed determination of diffuse radiation based on a anisotropic sky dome model is favorable. Herefore, the Perez model~\cite{Perez1986,Perez1987} is implemented.

\begin{dmath}
E_{d}(t,x) =  A(x) E_{d,h}(t) \left[ 0.5 \left(1+\cos i\right) \left(1-F_{1}\right) + F_{1} \cos\xi(t,x) \cos^{-1}\xi_{h}(t) + F_{2} \sin i\right]
\end{dmath}
\begin{dmath}
E_{r}(t,x) = 0.5 \rho A(x) \left(E_{D,h}(t) + E_{d,h}(t)\right) \left(1 - \cos i\right)
\end{dmath}

wherefore $A(x)$ is the surface area, $E_{d,h}$ is the diffuse horizonal radiation, $i$ is the surface inclination, $F_{1}$ is the circomsolar brightening coefficient, $F_{2}$, $\rho$ is the ground reflectance and $E_{D,h}$ is the direct horizonal radiation. The brightening coefficients $F_{1}$, $F_{2}$ are determined as

\begin{dmath}
F_{n} = F_{n,1}(\epsilon) + F_{n,2}(\epsilon) \Delta + F_{n,3}(\epsilon) \xi_{h}(t)
\end{dmath}

where $F_{n,i}$ are determined in the Perez model~\cite{Perez1987} based on measurements, $\epsilon$ is the sky clearness and $\Delta$ is the sky brightness. Here, the sky brightness $\Delta$ is determined as $E_{d,h} E_{sc}^{-1}$ while the sky clearness $\epsilon$ is determined as

\begin{dmath}
\epsilon = \left(E_{d,h}(t,x) + E_{D,h}(t,x)\right) E_{d,h}(t,x)^{-1} + \kappa \xi_{h}^{3}(t) \left(1 + \kappa \xi_{h}^{3}(t)\right)^{-1}
\end{dmath}

wherefore $E_{sc}$ is the solar constant and $\kappa$ equals 1.041.

\subsection{Solar shading}

So far, two types of structures which can shade a surface and thus reduce the shortwave radiation on a surface, i.e. an exterior screen in the pane of the surface and surface overhangs.

%\runinhead{Exterior solar screen.} 
The implementation of an exterior solar screen in the surface pane is stratightforward. The transmitted direct solar irradiation equals $E_{D}(t,x) \left(1 - f_{p}(t)\right)$ where $f_{p}(t)$ is the position of the screen between 0 and 1. The total transmitted diffuse and reflected solar irradiation equals $E_{d}(x,t) \left(1 - f_{p}\right) +f_{p}(t)  f_{s} E(x,t)$ where $f_{sha}$ is the shortwave transmittance of the exterior screen.

%\runinhead{Overhangs for vertical surfaces.} 
Horizontal and vertical projectionso above and besides windows are able to intercept the direct component of solar radiation depeneding on the geometry of the obstructing and receiving surface. The usnlit area $A_{SL}(x,t)$ of a surface with width $W$ and height $H$, and vertical and horizontal projections $P_{V}$ and $P_{H}$ at a distance $R_{W}$ and $R_{H}$ of the receiving surface edges is determiend as

\begin{dmath}
A_{SL}(x,t) = \left[W - P_{v}  \left|\tan \alpha(t,x) \right| + R_{W}\right] \left[H - P_{H} \tan \left(\pi - \xi_{h}(t)\right) \cos^{-1} \alpha(t,x) + R_{H}\right]
\end{dmath}

with $\alpha(t,x)$ the solar azimuth.

\bibliography{Climate/MyBibTexLibrary}

\end{document}

